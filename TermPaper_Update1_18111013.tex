\documentclass[12pt]{article}

\begin{document}

\title{Radiation Dosimetry in Nuclear Medicine}
\author{Ashish Saini (18111013)}
\date{January 21, 2022}
\maketitle




The treatment of a patient with a radioactive material termed a radiopharmaceutical that targets cancer or other aberrant, sick cells is known as radiopharmaceutical therapy (also known as molecular radiotherapy or "targeted" radionuclide therapy). Because the radiopharmaceutical selectively irradiates and destroys cancer cells while limiting radiation exposure to healthy tissue, this treatment is known as "targeted" radionuclide therapy. A radioactive atom (also known as a radionuclide) is paired with a targeting molecule that selectively seeks and binds to, in, or near cancer cells in radiopharmaceuticals. Without the use of a targeting molecule, some radionuclides can target specific cells. Because thyroid and thyroid cancer tissues naturally digest radioactive iodine, physicians have utilised radioiodine to safely treat hyperthyroidism (an overactive thyroid gland) and some types of thyroid cancer for many years. Because therapy may be adjusted to fit the molecular features of a patient's individual tumour, targeted radionuclide therapy holds promise for personalised treatment of many forms of cancer. Furthermore, because the radiopharmaceutical is delivered by the circulatory system, it can be used to treat tumours all throughout the body, including cancer that has spread to the point of being metastatic. On the other hand, surgery and external-beam radiation therapy (also known as teletherapy) can only remove malignancies from the treatment site, not from other parts of the body.
\par
Patients are given radionuclides in nuclear medicine treatments for a range of diagnostic and therapeutic purposes. The absorbed dosage to diverse organs of the patient is a critical consideration in such research; this concern is naturally heightened in therapeutic applications, where a large absorbed dose may be received by other organs, especially radiosensitive organs. The goal of this chapter is to examine the internal dosimetry methods and models used in nuclear medicine, as well as to address some current trends and issues in the area. It is not our objective to compile dose estimates for numerous nuclear medicine procedures; such compendia can be available in a variety of sources.






\end{document}