\documentclass[12pt]{article}

\begin{document}

\title{Radiation Dosimetry in Nuclear Medicine}
\author{Ashish Saini (18111013)}

\maketitle


\section{Introduction}

The treatment of a patient with a radioactive material termed a radiopharmaceutical that targets cancer or other aberrant, sick cells is known as radiopharmaceutical therapy (also known as molecular radiotherapy or "targeted" radionuclide therapy). Because the radiopharmaceutical selectively irradiates and destroys cancer cells while limiting radiation exposure to healthy tissue, this treatment is known as "targeted" radionuclide therapy. A radioactive atom (also known as a radionuclide) is paired with a targeting molecule that selectively seeks and binds to, in, or near cancer cells in radiopharmaceuticals. Without the use of a targeting molecule, some radionuclides can target specific cells. Because thyroid and thyroid cancer tissues naturally digest radioactive iodine, physicians have utilised radioiodine to safely treat hyperthyroidism (an overactive thyroid gland) and some types of thyroid cancer for many years. Because therapy may be adjusted to fit the molecular features of a patient's individual tumour, targeted radionuclide therapy holds promise for personalised treatment of many forms of cancer. Furthermore, because the radiopharmaceutical is delivered by the circulatory system, it can be used to treat tumours all throughout the body, including cancer that has spread to the point of being metastatic. On the other hand, surgery and external-beam radiation therapy (also known as teletherapy) can only remove malignancies from the treatment site, not from other parts of the body.
\par
Patients are given radionuclides in nuclear medicine treatments for a range of diagnostic and therapeutic purposes. The absorbed dosage to diverse organs of the patient is a critical consideration in such research; this concern is naturally heightened in therapeutic applications, where a large absorbed dose may be received by other organs, especially radiosensitive organs. 
\par
"Dosimetry" alludes to the science by which radiation not entirely settled by estimation, computation, or a mix of estimation and computation. The specialized name for radiation portion is "consumed portion"; it is how much radiation energy that is saved in tissue partitioned by the mass of the tissue. The consumed portion is the main actual element that impacts the reaction of cancers and the remainder of the body to radiation.
\par
Assimilated portion decides the degree to which cancers and typical tissues are impacted by radiation. The higher the consumed portion to growths, the more cells will be killed by radiation and the more noteworthy the probability of a fix. Be that as it may, the higher the consumed portion to typical tissues, the almost certain and extreme might be the unfortunate harmful results of the radiation. A significant benefit of radiopharmaceutical treatment is its capacity to light and viably treat growths all through the body; simultaneously, a few illumination of typical organs is unavoidable. Hence, the job of radiation dosimetry in designated radionuclide treatment is to decide explicitly, for every tolerant, the directed measure of the radiopharmaceutical that will most viably treat the patient's sickness while staying away from ingested dosages that harm typical tissues. Individualized radiation dosimetry is basic for arranging the best and most secure designated radionuclide treatment for every tolerant.
\par
The treating physician and the nuclear medicine technologist cooperate with medical physicists who specialise in procedures for determining absorbed dose. A tiny "dosimetry" amount of the therapeutic radiopharmaceutical may be given to a patient who undertakes a dosimetry study prior to receiving radiopharmaceutical therapy. After the dosimetry administration, the nuclear medicine technologist will acquire three or more nuclear medicine images at different periods. Only one image may be collected in some circumstances. Blood, urine, and stool samples may be taken by the treatment team. Based on the patient's unique bio distribution and clearance patterns from the body, the medical physicist analyses the dosimetry readings to compute an appropriate treatment for the therapy. Dosimetry guarantees that the patient receives a customised, safe, and successful treatment in this way.
\par
Patients are given radionuclides in nuclear medicine treatments for a range of diagnostic and therapeutic purposes. The absorbed dosage to diverse organs of the patient is a critical consideration in such research; this concern is naturally heightened in therapeutic applications, where a large absorbed dose may be received by other organs, especially radiosensitive organs. The goal of this chapter is to examine the internal dosimetry methods and models used in nuclear medicine, as well as to address some current trends and issues in the area. It is not our purpose to keep track of the amount of radiation emitted during many nuclear medicine operations.


\section{Theory}
\textbf{Internal dosimetry methods}
\begin{enumerate}
 \item  \textbf{Basic concepts} \par The absorbed dose in an organ can be calculated using the following formula: \par
	\begin{equation}
	D = {kA\sum nEf/m} 
	\end{equation}
	\par where D is the absorbed dose (rad or Gy), A is the cumulative activity (mCi h or MBq s), n is the number of particles with energy Ei emitted per nuclear 			transition, E is the energy per particle (MeV), f is the fraction of energy absorbed in the target, m is the mass of the target region (g or kg), and k is 				the proportionality constant (rad g/mCi h MeV or Gy kg/MBq s MeV).
 \item  \textbf{The MIRD system} \par In the Medical Internal Radiation Dose (MIRD) system, the absorbed dose equation is a deceptively simple formulation of Eq. (1): \par
	\begin{equation}
	D = AS
	\end{equation} \par
The total activity is indicated above, and all other terms are grouped together in the factor S: \par
	\begin{equation}
	S = {k\sum nEf/m} 
	\end{equation} \par
The standard factor k in the MIRD equation is 2.13, which calculates the absorbed dose in rad from activity in mCi, mass in g, and energy in MeV. With more applications using the SI unit system, a factor connecting absorbed dosage in Gy to activity in Bq and energy in MeV might be calculated and used.

\end{enumerate}

\section{Application}
\begin{enumerate}
\item \textbf{ The MIRD pamphlets} \par The Society of Nuclear Medicine's (MIRD) Committee has produced a number of useful papers and other tools for determining absorbed dose estimates in nuclear medicine applications. The first is a set of technical reports known as MIRD Pamphlets, which offer a wealth of information. Some booklets were left out, such as those that contained old deterioration data compilations. Many of the ones listed are still valuable since they offer information that isn't available anywhere else and is useful in many practical situations today (e.g. the pamphlets giving photon absorbed fractions for small objects). A series of studies detailing metabolic models and dose estimations for several radiopharmaceuticals is also available.

\item \textbf{ The proper design of kinetic studies} \par  A good internal dosage estimate is strongly reliant on the gathering of kinetic data for the organs that concentrate the radiopharmaceutical (source organs), the entire body, and all excretion paths. Obtaining these data necessitates the use of appropriate measuring methods and the collection of data at acceptable time intervals. The MIRD Committee has issued a paper that answers these concerns. The paper describes the necessary methodologies for data quantification and temporal sampling. They essentially demonstrate how to apply the conjugate view approach to gather quantitative data for dosimetry assessments, including the proper selection of source and background regions, as well as adjustments for overlapping source regions, background, and scatter. The application of SPECT and PET methods is also highlighted. The authors provide quantitative methods for evaluating blood and excreta samples. Concerning temporal sampling, they show that two or three time points per phase (either uptake or clearance) are required to fully represent the kinetics. They also depict graphically the amount of inaccuracy in A caused by ignoring an organ's wash-in phase or failing to correctly analyse the wash-out phase.

\item \textbf{ The use of EDE and ED} \par When a dosimetry study is successfully completed, the outcome is a collection of organ absorbed dose estimates, represented as total absorbed dose, based on an anticipated quantity of administered activity or absorbed dose per unit activity administered, as described in the MIRD System section above. This data will also provide the estimated effective dose equivalent (EDE) (ICRP, 1979) and effective dose (ED) (ICRP, 1991). The MIRDOSE programme also provides these settings automatically (Stabin et al.,1996). Individual organ absorbed dose estimations are the most important information to consider, especially in radionuclide treatment. When an absorbed dosage is met in the therapeutic arena, the EDE and ED have no relevance. They are valuable in diagnostics, comparing studies from various agents or from the same agent with different radionuclide labels, and assessing the population risk from specific investigations. However, it should be noted that these figures are hypothetical, based on risk weighting factors provided by a committee for various organs and radiobiological outcomes. These weighting variables are prone to change, as seen by the difference between ICRP 30's EDE and ICRP 60's ED.
\end{enumerate}

\end{document}