\documentclass[12pt]{article}
\usepackage{graphicx}
\graphicspath{ {./images/} }

\begin{document}

\title{Radiation Dosimetry in Nuclear Medicine}
\author{Ashish Saini (18111013)}

\maketitle

\section{Abstract} 
Radionuclides are utilized in atomic medication in an assortment of indicative and restorative methods. An information on the radiation portion got by various organs in the body is vital for an assessment of the dangers and advantages of any methodology. In this paper, current techniques for inward dosimetry are checked on, as they are applied in atomic medication. Especially, the Medical Internal Radiation Dose framework for dosimetry is made sense of, and a considerable lot of its distributed assets talked about. Accessible models addressing people of various age and orientation, including those addressing the pregnant lady are portrayed; latest things in it are additionally assessed to lay out models for individual patients. The appropriate plan of motor examinations for laying out radiation portions for radiopharmaceuticals is talked about. An outline of how to utilize data acquired in a dosimetry study, including that of the powerful portion same and successful portion, is given. Latest things and issues in inside dosimetry, including the computation of patient-explicit portions and in the utilization of limited scope and miniature dosimetry methods, are likewise looked into.

\section{Introduction}

The treatment of a patient with a radioactive material termed a radiopharmaceutical that targets cancer or other aberrant, sick cells is known as radiopharmaceutical therapy (also known as molecular radiotherapy or "targeted" radionuclide therapy). Because the radiopharmaceutical selectively irradiates and destroys cancer cells while limiting radiation exposure to healthy tissue, this treatment is known as "targeted" radionuclide therapy. A radioactive atom (also known as a radionuclide) is paired with a targeting molecule that selectively seeks and binds to, in, or near cancer cells in radiopharmaceuticals. Without the use of a targeting molecule, some radionuclides can target specific cells. Because thyroid and thyroid cancer tissues naturally digest radioactive iodine, physicians have utilised radioiodine to safely treat hyperthyroidism (an overactive thyroid gland) and some types of thyroid cancer for many years. Because therapy may be adjusted to fit the molecular features of a patient's individual tumour, targeted radionuclide therapy holds promise for personalised treatment of many forms of cancer. Furthermore, because the radiopharmaceutical is delivered by the circulatory system, it can be used to treat tumours all throughout the body, including cancer that has spread to the point of being metastatic. On the other hand, surgery and external-beam radiation therapy (also known as teletherapy) can only remove malignancies from the treatment site, not from other parts of the body.
\par
Patients are given radionuclides in nuclear medicine treatments for a range of diagnostic and therapeutic purposes. The absorbed dosage to diverse organs of the patient is a critical consideration in such research; this concern is naturally heightened in therapeutic applications, where a large absorbed dose may be received by other organs, especially radiosensitive organs. 
\par
"Dosimetry" alludes to the science by which radiation not entirely settled by estimation, computation, or a mix of estimation and computation. The specialized name for radiation portion is "consumed portion"; it is how much radiation energy that is saved in tissue partitioned by the mass of the tissue. The consumed portion is the main actual element that impacts the reaction of cancers and the remainder of the body to radiation.
\par
Assimilated portion decides the degree to which cancers and typical tissues are impacted by radiation. The higher the consumed portion to growths, the more cells will be killed by radiation and the more noteworthy the probability of a fix. Be that as it may, the higher the consumed portion to typical tissues, the almost certain and extreme might be the unfortunate harmful results of the radiation. A significant benefit of radiopharmaceutical treatment is its capacity to light and viably treat growths all through the body; simultaneously, a few illumination of typical organs is unavoidable. Hence, the job of radiation dosimetry in designated radionuclide treatment is to decide explicitly, for every tolerant, the directed measure of the radiopharmaceutical that will most viably treat the patient's sickness while staying away from ingested dosages that harm typical tissues. Individualized radiation dosimetry is basic for arranging the best and most secure designated radionuclide treatment for every tolerant.
\par
The treating physician and the nuclear medicine technologist cooperate with medical physicists who specialise in procedures for determining absorbed dose. A tiny "dosimetry" amount of the therapeutic radiopharmaceutical may be given to a patient who undertakes a dosimetry study prior to receiving radiopharmaceutical therapy. After the dosimetry administration, the nuclear medicine technologist will acquire three or more nuclear medicine images at different periods. Only one image may be collected in some circumstances. Blood, urine, and stool samples may be taken by the treatment team. Based on the patient's unique bio distribution and clearance patterns from the body, the medical physicist analyses the dosimetry readings to compute an appropriate treatment for the therapy. Dosimetry guarantees that the patient receives a customised, safe, and successful treatment in this way.
\par
Patients are given radionuclides in nuclear medicine treatments for a range of diagnostic and therapeutic purposes. The absorbed dosage to diverse organs of the patient is a critical consideration in such research; this concern is naturally heightened in therapeutic applications, where a large absorbed dose may be received by other organs, especially radiosensitive organs. The goal of this chapter is to examine the internal dosimetry methods and models used in nuclear medicine, as well as to address some current trends and issues in the area. It is not our purpose to keep track of the amount of radiation emitted during many nuclear medicine operations.


\section{Theory}
\textbf{Internal dosimetry methods}
\begin{enumerate}
 \item  \textbf{Basic concepts} \par The absorbed dose in an organ can be calculated using the following formula: \par
	\begin{equation}
	D = {kA\sum nEf/m} 
	\end{equation}
	\par where D is the absorbed dose (rad or Gy), A is the cumulative activity (mCi h or MBq s), n is the number of particles with energy Ei emitted per nuclear 			transition, E is the energy per particle (MeV), f is the fraction of energy absorbed in the target, m is the mass of the target region (g or kg), and k is 				the proportionality constant (rad g/mCi h MeV or Gy kg/MBq s MeV).
 \item  \textbf{The MIRD system} \par In the Medical Internal Radiation Dose (MIRD) system, the absorbed dose equation is a deceptively simple formulation of Eq. (1): \par
	\begin{equation}
	D = AS
	\end{equation} \par
The total activity is indicated above, and all other terms are grouped together in the factor S: \par
	\begin{equation}
	S = {k\sum nEf/m} 
	\end{equation} \par
The standard factor k in the MIRD equation is 2.13, which calculates the absorbed dose in rad from activity in mCi, mass in g, and energy in MeV. With more applications using the SI unit system, a factor connecting absorbed dosage in Gy to activity in Bq and energy in MeV might be calculated and used.

\end{enumerate}

\section{Application}
\begin{enumerate}
\item \textbf{ The MIRD pamphlets} \par The Society of Nuclear Medicine's (MIRD) Committee has produced a number of useful papers and other tools for determining absorbed dose estimates in nuclear medicine applications. The first is a set of technical reports known as MIRD Pamphlets, which offer a wealth of information. Some booklets were left out, such as those that contained old deterioration data compilations. Many of the ones listed are still valuable since they offer information that isn't available anywhere else and is useful in many practical situations today (e.g. the pamphlets giving photon absorbed fractions for small objects). A series of studies detailing metabolic models and dose estimations for several radiopharmaceuticals is also available.

\item \textbf{ The proper design of kinetic studies} \par  A good internal dosage estimate is strongly reliant on the gathering of kinetic data for the organs that concentrate the radiopharmaceutical (source organs), the entire body, and all excretion paths. Obtaining these data necessitates the use of appropriate measuring methods and the collection of data at acceptable time intervals. The MIRD Committee has issued a paper that answers these concerns. The paper describes the necessary methodologies for data quantification and temporal sampling. They essentially demonstrate how to apply the conjugate view approach to gather quantitative data for dosimetry assessments, including the proper selection of source and background regions, as well as adjustments for overlapping source regions, background, and scatter. The application of SPECT and PET methods is also highlighted. The authors provide quantitative methods for evaluating blood and excreta samples. Concerning temporal sampling, they show that two or three time points per phase (either uptake or clearance) are required to fully represent the kinetics. They also depict graphically the amount of inaccuracy in A caused by ignoring an organ's wash-in phase or failing to correctly analyse the wash-out phase.

\item \textbf{ The use of EDE and ED} \par When a dosimetry study is successfully completed, the outcome is a collection of organ absorbed dose estimates, represented as total absorbed dose, based on an anticipated quantity of administered activity or absorbed dose per unit activity administered, as described in the MIRD System section above. This data will also provide the estimated effective dose equivalent (EDE) (ICRP, 1979) and effective dose (ED) (ICRP, 1991). The MIRDOSE programme also provides these settings automatically (Stabin et al.,1996). Individual organ absorbed dose estimations are the most important information to consider, especially in radionuclide treatment. When an absorbed dosage is met in the therapeutic arena, the EDE and ED have no relevance. They are valuable in diagnostics, comparing studies from various agents or from the same agent with different radionuclide labels, and assessing the population risk from specific investigations. However, it should be noted that these figures are hypothetical, based on risk weighting factors provided by a committee for various organs and radiobiological outcomes. These weighting variables are prone to change, as seen by the difference between ICRP 30's EDE and ICRP 60's ED.

\item \textbf{  Uses of absorbed dose information} \par The data acquired in a dosimetry study is utilized in a wide range of ways, including assessment of individual preliminaries, and in the endorsement of radioactive medications for general use. Radiation portion gauges for individual organs, typically for the a few organs getting the most elevated portion, and the EDE or ED, on account of indicative investigations, are utilized to assess the radiation portion expected to be gotten, and hence the greatest measure of movement that ought to be controlled. Clearly in remedial circumstances, the assessment is more significant, as the radiation portion got is a lot higher. There are a few areas of dynamic examination wherein upgrades in this data are being thought of. The first, as talked about straightaway, is the adjustment of current strategies to give radiation portion gauges that are more explicit to the subject viable, instead of the portrayal of all patients by the normalized subjects examined in the segment on Phantoms, above. Another issue that numerous analysts are researching is that revealed radiation portion to the red marrow frequently doesn't connect well with noticed deterministic radiation impacts saw in treatment patients. This is probably because of contrasts between the individual and the standard apparitions utilized, in size and shape, yet in addition in the wellbeing and appropriation of the marrow. Most people associated with restorative preliminaries have some type of illness which might influence marrow dissemination, however too may have gotten different types of treatment beforehand (for example chemotherapy, limited outside radiotherapy). In this manner their red marrow might have been impacted essentially, making it be significantly not quite the same as that in the standard model, both as to conveyance and reasonability. A third area of concern presently is the computation and translation of radiation portion to tissues or little constructions not generally perceived as ''organs''. It is conceivable with current techniques to compute consumed portion to constructions of practically any size or shape.
\par Particular take-up of radiopharmaceuticals and the subsequent radiation portion, have been contemplated in the lachrymal organ, salivary organs and in little constructions inside the mind and the eye. Also, organs which have been customarily treated as uniform in sythesis may as a matter of fact include districts inside themselves that require separate assessment as far as radiation portion, for example, the cortex and medulla of the kidneys. Despite the fact that it is feasible to work out this portion with great exactness, the understanding of this data from a security or administrative viewpoint isn't grounded.
\end{enumerate}

\section{Current trends}
\begin{enumerate}
\item \textbf{Patient-specific dosimetry} \par Radionuclide treatment in light of patient-explicit dosimetry others the potential for improving the portion conveyed to the objective cancer through usage of estimated radiopharmaceutical energy explicit to the person. The controlled movement might be custom fitted for the patient with the end goal that the most elevated conceivable radiation portion might be given to the growth while restricting the portion to basic organs and tissues beneath any assigned edge for negative natural impacts. \par Preferably, the objectives for treatment arranging methodology utilizing patient-explicit dosimetry are: \par
	\begin{enumerate}
		\item Procure time sequenced quantitative information utilizing indicative exercises of the helpful radiopharmaceutical (or reasonable simple) to decide 			the biodistribution throughout the pertinent time-course for that specialist. This might be accomplished through consecutive imaging utilizing a 					planar gamma camera or tomographic frameworks, for example, single photon outflow figured tomography or position emanation tomography. 
		\item	Gauge the radiation portion to the cancer and other objective organs (basic tissues) per unit regulated action utilizing these patient-explicit 						biokinetic information inside the MIRD calculational pattern portrayed beforehand. SPECT and PET have the potential for giving 3-layered 						informational indexes which may be utilized straightforwardly with Monte Carlo and other scientific calculations to create 3D consumed portion 					maps. In these circumstances, it is useful to utilize X-beam figured tomography or attractive reverberation imaging to give lessening remedy 					techniques to the radionuclide checks and the physical base for the 3D retained portion maps.	
		\par
		\includegraphics{flow chart} \par  \begin{center} Fig-1: Stream diagram depicting the fundamental stages in a patient explicit dosimetry convention. 										\end{center} \par
		\item Foresee the radiation portion to be conveyed to the patient under the treatment routine through extrapolation of the indicative portion results 					scaled by the regulated exercises. In any case, it should likewise be perceived that biokinetics for the indicative and treatment managed exercises 			probably won't be indistinguishable as could happen assuming the symptomatic action is suficient to incite the so-call 'staggering' impact on the 					objective tissue consequently diminishing ensuing take-up. This impact has been seen on account of demonstrative work-ups for thyroid disease 					while utilizing higher symptomatic exercises.	
		\item Screen the radiopharmaceutical biokinetics during treatment for correlation with the analytic forecast. Stream outline portraying the fundamental 			stages in a patient explicit dosimetry convention. This fulfills the scholastic interest including questions concerning the connection of the 						demonstrative and treatment energy and permits check of the genuine helpful portion conveyed.
		\item Assess the adequacy of the treatment following treatment to anticipate and even keep away from future potential complexities in different 						patients. For this, portion volume histograms (a portrayal of the appropriation of consumed portion in histogram design), growth control 						likelihood and ordinary tissue entanglements likelihood ought to be thought about.		
		
	\end{enumerate}
\item \textbf{Absorbed dose calculation} \par For homogeneous action circulation in organs, MIRD S values have been utilized in both indicative and treatment
assimilated portion computations. Program bundles have been proposed; as verified above in the primary case, the incorporation of cancers is conceivable. The biggest single wellspring of blunder in the method is frequently in the biokinetics, particularly the vulnerability in the late movement time information. Nonetheless, in circumstances in which the mass of the objective locale is hard to decide, this might present the biggest wellspring of blunder. The utilization of S values in view of normalized people, regardless of whether scaled utilizing the genuine organ mass from patient explicit information, may likewise bring huge blunders into the investigation. On the off chance that the movement circulation is generally uniform inside the organ, the standard approximations might be somewhat great, however assuming there are significant inhomogeneities (the presence of a hot or cold cancer, and so on), estimations in light of the suspicion of a uniform action conveyance might be essentially in blunder. \par
Blood inspecting, pee and excrement assortment give extra data on the action appropriation. In radionuclide treatment, particularly radioimmunotherapy, other than realizing the action content in cancers and tissues, the assimilated portion dissemination is required for appropriate assessment of the remedial adequacy. For such estimations film autoradiography might be utilized. In spite of the fact that it is slow, it has an excellent goal (from 10±100 mm). Film may likewise be supplanted by the a lot quicker fluorescent plates, which have goals between 50±150 mm. This technique has the weakness of a failure to acquire ongoing pictures, consequently forestalling direct client controlled examining. Likewise, only one radionuclide can be imaged at the same time because of the absence of energy goal. Strong state indicators, for example, the silicon beta camera are additionally a work in progress. Involving a high goal identifier for imaging and direct digitization, a 3D picture framework of the action circulation might be made, which thusly might be utilized to acquire a 3D ingested portion appropriation.

\item \textbf{ Small scale dosimetry and microdosimetry} \par At the point when the take-up of a radiopharmaceutical in an objective tissue is especially nonuniform, the averaging of the portion over the whole tissue might be a misrepresentation of the genuine energy statement design. To give proper data to understanding of the portion, the itemized energy testimony design over short (for example cell) reaches might should be examined. This is particularly significant for short reach radiations (for example low-energy photons, inner change electrons(ICE), Auger electrons (AE), Coster-Kronig electrons) which might be ingested inside a brief distance. The work with mouse testicles exhibited that Tl-201, which radiates overwhelmingly low energy mercury X-beams, AEs and ICEs, was two to multiple times more powerful in decreasing testis weight and sperm number than Tl-204 (a high energy beta producer) per unit movement managed.

\end{enumerate}

\section{Conclusion}
This article has attempted to summarize the basic
concepts of internal dosimetry in which we have discussed about the basic principles and the MIRD system and follow its application such as MIRD pamphlets, use of EDE and the uses of absorbed dose information. In the sequence then we discussed about the current development up to recent years in various fields, such as patient specific dosimetry, absorbed dose calculation and small scale dosimetry and microdosimetry, simulation of particle interactions with matter and computational techniques. Apart from diagnostic imaging, using radionuclides in nuclear medicine for therapy purposes has increased the requirement for accuracy in calculating absorbed dose. 

\section{References}
\begin{enumerate}
\item Stabin, M., and Xu, X. G. (2014). Basic Principles in the Radiation Dosimetry of Nuclear Medicine. Seminars in Nuclear Medicine, 44(3), 162–171. doi:10.1053/j.semnuclmed.2014.03.
\item https://dceg.cancer.gov/tools/radiation-dosimetry-tools/nuclear-medicine.
\item Stabin, M. (2006). Nuclear medicine dosimetry. Physics in Medicine and Biology, 51(13), R187–R202. doi:10.1088/0031-9155/51/13/r12.
\item Zanzonico, Pat B. The Journal of Nuclear Medicine; New York Vol. 41, Iss. 2,  (Feb 2000): 297-308.
\item Radiation Protection Dosimetry, Volume 165, Issue 1-4, July 2015, Pages 416–423, https://doi.org/10.1093/rpd/ncv061.
\item A. B. Brill; M. Stabin; A. Bouville; E. Ron Radiat Res (2006) 166 (1): 128–140. https://doi.org/10.1667/RR3558.1
\item Hubert Vesselle, John Grierson, Lanell M. Peterson, Mark Muzi, David A. Mankoff and Kenneth A. Krohn Journal of Nuclear Medicine September 2003, 44 (9) 1482-1488;
\item Salvatori, M., Cremonesi, M., Indovina, L., Chianelli, M., Pacilio, M., Chiesa, C.,  Zanzonico, P. (2017). Radiobiology and Radiation Dosimetry in Nuclear Medicine. Nuclear Oncology, 305–349. doi:10.1007/978-3-319-26236-9 6
\item Srinivasan Senthamizhchelvan, Paco E. Bravo, Martin A. Lodge, Jennifer Merrill, Frank M. Bengel and George SgourosJournal of Nuclear Medicine March 2011, 52 (3) 485-491; DOI: https://doi.org/10.2967/jnumed.110.083477 
\end{enumerate}

\end{document}